% Cover letter using letter.sty
\documentclass{letter} % Uses 10pt
%Use \documentstyle[newcent]{letter} for New Century Schoolbook postscript font
% the following commands control the margins:
\topmargin=-1in    % Make letterhead start about 1 inch from top of page 
\textheight=10in  % text height can be bigger for a longer letter
\oddsidemargin=0pt % leftmargin is 1 inch
\textwidth=6.5in   % textwidth of 6.5in leaves 1 inch for right margin

\begin{document}

\signature{Jack Blumenau}           % name for signature 
\longindentation=0pt                       % needed to get closing flush left
\let\raggedleft\raggedright                % needed to get date flush left
 
 
\begin{letter}{Department of Government \\
London School of Economics}


\begin{flushleft}
{\large\bf Jack Blumenau, PhD. Candidate, London School of Economics}
\end{flushleft}
\medskip\hrule height 1pt
\begin{flushright}
\hfill 07742 506757, j.blumenau@lse.ac.uk, 29c Marquis Road, Finsbury Park, London, N4 3AN 
\end{flushright} 
\vfill % forces letterhead to top of page

 
\opening{Dear Sir or Madam,} 
 
\noindent I would like to be considered for the position of LSE Fellow in Government (GV101).
 
\noindent Over the past year I have taught extensively for the LSE's Methodology Department, teaching both introductory and advanced courses in statistical analysis. I received extremely positive teaching evaluations for both of these courses, and feel that my ability to produce clear, accessible, and accurate interpretations of complex material is one of my major strengths. I developed strong relationships with both the students and my fellow teachers on these courses, and have continued to mentor a number of students in their MA theses. I relish the prospect of teaching an introductory course in political science, which would allow me to combine my substantive interest in politics with the teaching skills I have developed over the past year.

\noindent My personal research uses advanced quantitative methods to investigate comparative questions of legislative politics, party competition, and electoral systems. In my PhD thesis I focus on the voting and speech-making behaviour of legislators in the European Parliament. My first paper, which was awarded the Wildenmann Prize for best paper presented at the ECPR Joint Sessions of workshops in 2012, explores how party leaders in the European Parliament restrict access to the plenary agenda so as to preserve party unity on roll call votes. In the second paper, I use a Bayesian ideal-point model to estimate the effect of the financial crisis on voting coalitions in the European Parliament. In my final paper, I have collected a new dataset of a quarter of a million legislative speeches and employ recent advances in quantitative text analysis to estimate the `expressed agenda' of MEPs. I intend to submit my PhD in March 2016.

I also have working papers on British electoral competition, the effects of information provision on strategic voting in close elections, and the arithmetic of legislative coalitions. A recent co-authored paper of mine - on the effects of a switch to an open-list proportional representation system for UK European elections - was recently accepted for publication in the prestigious British Journal of Political Science. More generally, my work covers a wide range of topics included in the GV101 syllabus, and makes me ideally suited to teaching these topics to undergraduates.

\noindent As my CV makes clear, I am highly proficient in statistical analysis and interpretation. I am well versed in multivariate methods, and have demonstrated my ability to communicate the intuition and detail of these methods to students through my teaching experience in the Methodology Department. My PhD work uses these methods extensively, meaning that I have a strong command of the relevant techniques and methods that form a core part of the GV101 curriculum.

Additionally, I have also been employed for the past year as the head of the LSE's British general election coverage for the 2015 election. In addition to helping to develop a statistical model to predict the election outcome, I have written extensively for the LSE's election blog (which I also manage) on a variety of topics pertaining to the election. These blog posts explain complex political science theories and results to a non-technical audience. Presentation and communication skills have been at the heart of my work on this project, and much of my work has featured in the New Statesman, the Guardian, the Financial Times, and on the BBC's Newsnight. Again, I feel that this experience has put me in a good position to explain similar concepts, theories, and results to an undergraduate audience.
 
\noindent I would appreciate your consideration of my application for this position.
 
\closing{Yours sincerely,} 
 

 


\end{letter}
 

\end{document}






