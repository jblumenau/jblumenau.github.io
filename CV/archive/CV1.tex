% Cover letter using letter.sty
\documentclass{letter} % Uses 10pt
%Use \documentstyle[newcent]{letter} for New Century Schoolbook postscript font
% the following commands control the margins:
\topmargin=-1in    % Make letterhead start about 1 inch from top of page 
\textheight=10in  % text height can be bigger for a longer letter
\oddsidemargin=0pt % leftmargin is 1 inch
\textwidth=6.5in   % textwidth of 6.5in leaves 1 inch for right margin

\begin{document}

\signature{Jack Blumenau}           % name for signature 
\longindentation=0pt                       % needed to get closing flush left
\let\raggedleft\raggedright                % needed to get date flush left
 
 
\begin{letter}{Department of Government \\
London School of Economics}


\begin{flushleft}
{\large\bf Jack Blumenau, PhD. Candidate, London School of Economics}
\end{flushleft}
\medskip\hrule height 1pt
\begin{flushright}
\hfill 07742 506757, j.blumenau@lse.ac.uk, 29c Marquis Road, Finsbury Park, London, N4 3AN 
\end{flushright} 
\vfill % forces letterhead to top of page

 
\opening{Dear Sir or Madam,} 
 
\noindent I would like to be considered for the position of Polling Research Assistant for British Government at the LSE.
 
 \noindent My personal research interests focus on legislative politics, party competition, electoral systems and, more generally, the application of quantitative methods to questions of comparative politics. In my PhD thesis I focus on the voting and speech making behaviour of Members of the European Parliament (MEPs).  I have three papers, all of which use advanced quantitative methods to explore questions of legislative politics. My first paper, which was awarded the Wildenmann Prize for best paper presented at the ECPR Joint Sessions of workshops in 2012, explores how party leaders in the European Parliament restrict access to the plenary agenda so as to preserve party unity on roll call votes. In the second paper, co-authored with Benjamin Lauderdale, we use a Bayesian ideal-point model to estimate the effect of the financial crisis on voting coalitions in the European Parliament. In my final paper, I have collected a new dataset of a quarter of a million legislative speeches and employ recent advances in quantitative text analysis to estimate the `expressed agenda' of MEPs, which I link to constituency level public opinion. I hope to submit all three of these papers for publication over the next six months.
 
\noindent As my CV makes clear, I am proficient in all the technical skills required for this position. I am well versed in the R programming language, and have a great deal of experience in integrating and evaluating datasets, and creating meaningful figures. My PhD work uses public opinion data extensively, meaning that I have a strong command of the relevant techniques and methods used in the analysis and evaluation of this data. Beyond the technical, I have a great deal of experience developing statistical models, and feel that I can contribute creative solutions to statistical and other methodological problems. 

Additionally, I am confident in the quality of my written work, and I enjoy conveying topical findings to a variety of audiences - academic and otherwise. As a PhD student at LSE I co-authored a recent paper (currently under review) with Simon Hix, Andrew Eggers, and Dominik Hangartner which received attention not only from within the academic community, but also in wider policy-making circles. I produced a blog summarising this research which was featured on the LSE British Politics and Policy website (details of which can be found on my CV). I thoroughly enjoyed this experience, and feel that my ability to produce clear, accessible, and accurate interpretations of complex material is one of my major strengths.

I enjoy working in a team, am able to work productively upon my own initiative. I am confident in my ability to forge and maintain strong working relationships with other academics and with other parties, such as polling companies. During my PhD, I have worked extensively with members of the Government and Methodology departments at LSE and I would relish the opportunity for more extensive contact with those in both departments.
 


\noindent I would appreciate your consideration of my application for this position.
 
\closing{Yours sincerely,} 
 

 


\end{letter}
 

\end{document}






